
\chapter{Introduction}
 \epigraph{Every moment of consciousness is a precious and fragile gift.}{\textit{Steven Pinker}}

\section{Prospective Memory error}


Have you experienced when you wake up from your bed in the morning, put your glasses on and
go to the kitchen to get a glass of milk. But when you are in the kitchen, you forget what
you intended to do. This phenomenon is called prospective memories failures.

Prospective memory is the ability in the future to remember to do an action that previously planned without being instructed to do so \citep{GROOT2002}. This type of memory is different with retrospective memory which is the memory that we use when we are answering a question in the exam. Retrospective memory involves remembering event, words, and so on from the past typically when deliberating to do so.

Prospective memory failures are common in everyday life, almost 50\% of forgetting in our daily routines are due to of prospective
memory error \citep{Crovitz1984}. This memory failure can lead to embarrassment such as forget that you had arranged a meeting with your friend and even result in
serious injury or death. One example of a horrible case is "After a change in his usual routine; an
adoring father forgets to turn toward the daycare center and instead drove his usual route to work
at the university. Several hours later, his infant son, who had been quietly asleep in the back seat,
was dead " \citep{Einstein2005}. So it is important to have a great understanding about prospective memory error.

But what makes us forget ?. \cite{Radvansky2006} and \cite{Radvansky2010} shows that if people make a transition from one event to another, for examples move
from one room to another room, they tend to forget more information than if they do not.
\cite{Cockburn1994} Show that stress and anxiety cause us to become absent-minded and
thus produce failures of prospective memory. There is also a lot of study about ageing and its relation to prospective memory, one of it is a study conducted by \cite{Scullin2012} found older people tend to make more error than younger people on a prospective memory test.

The purpose of this MSc Dissertation project is to build an application that can use to conduct an experiment about prospective memory error, analyse the effect of multiple intention on a failure of prospective memory and to make a further understanding of what happens
during event boundary (e.g., moving to another application inside the smartphone) by tracking the activity of the participant during the prospective memory task.
The experiment conducted on this thesis is originally based on studies done by Carlson (what Did I Come here to do ?, Pennsylvania State University 2016).


\section{Project goals}
The main goals of the thesis are to create an application that can be used to other researchers to conduct a prospective memory experiment.
The application should able to conduct three type of studies from Carlson's experiment.
Three studies are conducted to analyze the influence of multiple intentions (is attentional loads matter?) and event boundaries (event horizon model)
on prospective memory error.

\section{Structure of dissertation}

The document is structured as follows
\begin{itemize}
\item In the \textbf{Literature Review} chapter provides an explanation about the prospective memory, retrospective memory and what influence the prospective memory error.
The different element of prospective memory is explained here.

\item In the \textbf{Experiment and Application Design} chapter provides an information about the architecture and the design of the experiment and the application.
How the experiment is conducted and its properties are explained. The main flow and the user design of the application are also provided.

\item In the \textbf{Implementation} chapter provides information about the technical implementation of the experiment application based on the design and the requirement.
 This chapter explains how the flow of the application works and how the features is implemented.

\item In the \textbf{Experiment result and Discussion} chapter provides the result and analysis of the output of the experiment.

\item in the \textbf{Conclusion and Suggestion} highlight the summary and achievement of the application and experiment. And also giving an opinion about possible future improvement and research.


\end{itemize}
