q
\chapter{Experiment Result and Discussion}

This chapter provides the result and the analysis of the experiment. three hypothesis are analyzed;
\begin{itemize}
  \item{Do participant experience the failure of prospective memory while using the smartphone?}
  \item{Is failure of prospective memory is more likely to happen with two intentions rather than one; intentional load matters}
  \item{What is the effect of notification to prospective memory error}
  \item{Does mentaly moving through event boundary increase the likeliness to experience failure of prospective memory?}
\end{itemize}

\section{Prospective memory error on smartphone}

\subsection{Experiment result}
In these expriment the data from the participants from the three studies is combined.
Table \ref{fig:affirmationTable} shows who many participants believe that they have experienced prospective memory error,
and how many participant actually experience the prospective memory error during the experiment.
The participant were asked if they believe that they had experince prospective memory error using the first question on the Table \ref{tab:demographicQuestion}.
Actual experience of memory error was calculated by looking if the people forget the questions during the experiment.

%The prospective memory error on the experiment is presented by the lost of intention during the study.
% The first column shows the affirmation on the prospective memory. the number is obtained by asking the participants \textit{"Often people go into a room to do something.  Though they know they intended to do something, they lose track of what they wanted to do.
% This same sort of thing can happen when using a smart phone, as well.  During the study,
% you may have clicked on a link, gone to the website, and then forgot what you intended to look up.  Did that happen to you at all during this study?"} on the post question phase during the experiment.
% While the second column is obtained by counting how many participant forget the question and chosed to look at the question again (lookback) during an experiment.
% The last column shows the total number of the participant participated on each study.
 %or forget the question during the experiment.

\subsection{Discussion}
% As seen in figure \ref{fig:demo1Study1}, \ref{fig:realdemo1Study1}, \ref{fig:demo1Study2}, \ref{fig:realdemo1Study2}, \ref{fig:demo1Study3}, \ref{fig:realdemo1Study3}, the left
% pie chart shows percentage of participant who think they are experiencing prospective memory error, and on the right side shows the percentage of participant who experience
% the prospective memory failure during the experiment.

Most of the participant did not believe that they have experienced the prospective memory error
. In contrast, the output shows that  almost 70\% of the participant actually experienced the prospective memory error.
We can agrue that the participant made an intention for looking the answer before clicking the answer link, but after reading the answer page
they lost their original intention.
As a result they forget the content of the question, and they experience prospective memory error.
This shows that while using a smartphone a person has a high probability of experiencing prospective memory error.
This experiment support the result of Prof. Richard alan Carlson's experiment.

% \begin{table}[]
% \centering
% \label{my-label}
% \begin{tabu}{|X[3,c]|X[3,c]|X[3,c]|}
% \hline
% \multicolumn{3}{|c|}{Study 1}                                                                             \\ \hline
% Affirmation of prospective memory error & \begin{tabular}[c]{@{}l@{}}Forget the questions \end{tabular} & Total Person \\ \hline
% 0                      & 3                                                                 & 4            \\ \hline
% \multicolumn{3}{|c|}{Study 2}                                                                             \\ \hline
% Affirmation of prospective memory error & Forget the questions  & Total Person \\ \hline
% 5                      & 8                                                                 & 11           \\ \hline
% \multicolumn{3}{|c|}{Study 3}                                                                             \\ \hline
% Affirmation of prospective memory error & Forget the questions                                            & Total Person \\ \hline
% 2                      & 2                                                                 & 3            \\ \hline
% \end{tabu}
% \caption{Participant affirmation and their experiment result on prospective memory error on smartphone}
% \label{fig:affirmationTable}
% \end{table}


\begin{table}[]
\centering
\small
\footnotesize
\begin{tabu}{|X[6,l]|X[2,c]|X[2,c]|X[2,c]|}
\hline
                                                                           & Experiment 1 (n=4) & Experiment 2 (n=11) & Experiment 3 (n=3) \\ \hline
How many participant believe they have experience prospective memory error & 0                  & 5                   & 2                  \\ \hline
How many participant actually experienced prospective memory error          & 3                  & 8                   & 2                  \\ \hline
\end{tabu}
\caption{Number of participant from all the studies who believe they have experince prospective memory error and the actual result of the experiment}
\label{fig:affirmationTable}
\end{table}



\section{The effect of multiple intention}

\subsection{Experiment result}
This section shows the result from the second study. On the second study, one or two question are presented to the participant randomly.
A participant intention is to look for the answer. Thus the number of question presented is the number of intention need to be retained.
Using the result we are trying to see  if increasing the number of intention it will make people more likely to experience failure of prospective memory (is the intentional loads matter ?).
Table \ref{fig:oneTwoQuestionForget} shows the total number of times the participant forget the question on the experiment.
It shows that when the participant presented with two question they are 75\% more probably to forget the
question rather than presented by one question.

Figure \ref{fig:TTFA_oneTwoQues} shows how long in millisecond the participants need to write the answer (TTLFA) of each question if one question and if two questions are presented.
The horizontal axis shows the 11 participant and the vertical axis shows the duration of writing. It shows that 63\% (7 out of 11) have longer time
writing the answer if two questions are presented each time. but the remaining two participant have more or less similar time on writing the answer both on
one or two questions presented.
In general figure \ref{fig:TTLFA_oneTwoQuesGeneral} shows how the average writing time on one or two question for all participant in the second study.
It also shows that if two questions are presented then it will take longer time for the people to recall the answer and write the answer.

Figure \ref{fig:lookingAnswer_oneOrTwo} shows the average time each participant spent on looking for the answer on the answer page. The top plot descibe the average time when
the participant look at the question first time. Surprisingly it shows that almost half of the particpants (4 out of 11) spent significantly longer time to look at the answer for one rather than two questions.
The lower plot shows the time spent looking for the answer after the participant look at the question again (lookback). It shows that most of the participant forget more frequenlty the
question and spent longer time on looking for the answer if two questions are presented at each time.

\subsection{Discussion}

The result of this study shows that the amount of intentional loads are important component on prospective memory. Based on the table \ref{fig:oneTwoQuestionForget}, a person is more probable to experience
prospective memory error if the amount of the intention is higher.

Furthermore, the result on figure \ref{fig:TTFA_oneTwoQues} and figure \ref{fig:TTLFA_oneTwoQuesGeneral} shows that the increasing amount intentional loads also make the person harder to recall the content of the intention. On this analysis,
this intention is different with the first intention which looking for the answer, but the intention
is to answer the question and it's formed after the participant found the answer on the answer page. the recall time is presented as the time participant write the answer.

Based on figure \ref{fig:lookingAnswer_oneOrTwo}, the increasing amount of intention also increase the time the participant spent on looking for the answer.
Moreover, when the failure of prospective memory occurs and the person need to look the question again, the time they spent looking for the answer on the two intentions is higher than one intention.
these result shows that the number of intention decrease their cognitive performance which result on the likeliness of experiencing the failure of prospective memory.
This probably because the increase of intentions will reduce the level of attention given on the task\citep{Reason1984}, and make
the participant take a longer time to find the answer.


% Please add the following required packages to your document preamble:
% \usepackage{multirow}
\begin{table}[!h]
\centering
\begin{tabular}{|l|l|l|}
\hline
\multirow{2}{*}{Participant} & \multicolumn{2}{l|}{How many times the participant forget the question} \\ \cline{2-3}
                             & One question                  & Two question                 \\ \hline
1                            & 0                             & 1                            \\ \hline
2                            & 1                             & 2                            \\ \hline
3                            & 0                             & 2                            \\ \hline
4                            & 1                             & 0                            \\ \hline
5                            & 0                             & 0                            \\ \hline
6                            & 0                             & 0                            \\ \hline
7                            & 1                             & 0                            \\ \hline
8                            & 0                             & 1                            \\ \hline
9                            & 0                             & 0                            \\ \hline
10                           & 0                             & 2                            \\ \hline
11                           & 1                             & 2                            \\ \hline
Sum                          & 4                             & 10                           \\ \hline
\end{tabular}
\caption{The number of question the participants forget}
\label{fig:oneTwoQuestionForget}
\end{table}

% Hampir 75\% dari kelupaan adalah two questions.
%
% then show that TTLFA time they write the answer also longer
% then show that using two question make people look at the answer longer time,
% turns out not. but most of the time participant look again the answer for two question

\begin{figure}[!h]
\centering
\begin{minipage}{.5\textwidth}
  \centering
  \includegraphics[scale=0.5]{TTFLA_each_participant}
  \captionsetup{justification=centering}
  \captionof{figure}{Average time each participant filling the answer}
  \label{fig:TTFA_oneTwoQues}
\end{minipage}%
\begin{minipage}{.5\textwidth}
  \centering
  \includegraphics[scale=0.5]{TTLFA_general}
  \captionsetup{justification=centering}
  \captionof{figure}{Average time of all the participants filling the answer}
  \label{fig:TTLFA_oneTwoQuesGeneral}
\end{minipage}
\end{figure}


\begin{figure}[!h]
\begin{center}
\includegraphics[scale=0.86]{visitedLink_each_participant}
\end{center}
\captionsetup{justification=centering}
\caption{Average time in spent looking for an answer between one or two questions}
\label{fig:lookingAnswer_oneOrTwo}
\end{figure}


% \begin{table}[]
% \centering
% \caption{My caption}
% \label{my-label}
% \begin{tabular}{|l|l|l|l|l|l|l|l}
% \cline{1-7}
% \multirow{2}{*}{Question}           & \multicolumn{2}{l|}{Study 1} & \multicolumn{2}{l|}{Study 2} & \multicolumn{2}{l|}{Study 3} &  \\ \cline{2-7}
%                                     & Yes           & No           & Yes           & No           & Yes           & No           &  \\ \cline{1-7}
% Experience prospective memory error & 0             & 4            & 5             & 6            & 2             & 1            &  \\ \cline{1-7}
% Forget about the answer             & 2             & 2            & 4             & 7            & 0             & 3            &  \\ \cline{1-7}
% \end{tabular}
% \end{table}

% Please add the following required packages to your document preamble:
% \usepackage{multirow}


\subsection{The effect of notifiaction on the intention}
\subsection{Experiment Result}
Table \ref{tab:notifiactionNumber} shows the time participant spent on writing the answer (TTLFA), average time looking for the answer
and the percentage of lookback on each number of notification shown up.
It show that by increasing the number of notification people spent more time writing the answer and looking for the answer.

\begin{table}[]
\centering
\begin{tabular}{|l|l|l|}
\hline
\multicolumn{3}{|c|}{No Notification}                      \\ \hline
TTFA    & Average time looking  at answer page & Lookback percentage \\ \hline
6267.68 & 59875.22 millisecond                   & 17\%                   \\ \hline
\multicolumn{3}{|c|}{One Notifications}                     \\ \hline
TTLFA   & Average time looking at answer page  & Lookback frequency \\ \hline
7292.87 & 69517.0 millisecond                    & 11\%                 \\ \hline
\multicolumn{3}{|c|}{Two Notifications}                      \\ \hline
TTLFA   & Average time looking at answer page  & Lookback frequency \\ \hline
7304.87 & 76699.16  millisecond                   & 21\%                 \\ \hline
\end{tabular}
\caption{The experiment result on each notification number}
\label{tab:notifiactionNumber}
\end{table}

\subsection{Disscussion}
Table \ref{tab:notifiactionNumber} shows that notification and increasing the number of notification make people harder to recall the content of the memory, represented by the average time participant looking for the answer.
We can argue that the notification probably make the level of attention of the participant lower thus make the intention is not framed perfectly
which make the participant experience the detached intention \citep{Reason1984}. Since the notification
also occurs while the participant looking at the answer, interestingly this probably can also mean that the notification distract the intention, even though it's correctly framed before.
Optionaly, the attention probably make the new event model and make people experience new event boundary since the more we present it to the
participant the more they experience harder they recall the content of the intention.

In addition, the notifiaction can also be considered as an event boundary. However, table \ref{tab:notifiactionNumber} shows
that the notification showed up and its quantity is not linear with the percentage of forgetting the question.
So it shows that mentaly moving through event boundary will have no effect on the prospective memory error.

\subsection{Event boundary on prospective memory}
\subsection{Experiment Result}
The bar chart \ref{fig:lookingAnswer_lookback} shows the average time (in millisecond) the participants in all the studies spent looking for the answer on the answer page on each questions.
The chart shows that if the participant forget the question and decide to see it again (lookback), they spent longer time looking for the answer compare to if they don't forget the question(non-lookback).
The bar chart \ref{fig:freq_study1} and \ref{fig:freq_study3} shows the frequency of looking the question again (lookback) for all questions  between first and the third study.
It shows that the peole more frequently do a lookback on third study rathern than on study one.
The bar chart \ref{fig:aveTime_study1} and \ref{fig:aveTime_study3} shows the time spent looking for the answer for all the questions between first and the third study.
It shows that the participant spent longer time to look at the answer and they spent longer time to look at the answer again after the do lookback.
\begin{figure}[!h]
\begin{center}
\includegraphics[scale=0.86]{lookback_and_reading_time_studyAll}
\end{center}
\captionsetup{justification=centering}
\caption{Average time in spent looking for an answer between lookback and non-lookback in all studies}
\label{fig:lookingAnswer_lookback}
\end{figure}

\begin{figure}[!h]
\centering
\begin{minipage}{.5\textwidth}
  \centering
  \includegraphics[width=\textwidth]{frequency_lookback_study1}
  \captionsetup{justification=centering}
  \captionof{figure}{Frequency of lookback of the participant on study 1}
  \label{fig:freq_study1}
\end{minipage}%
\begin{minipage}{.5\textwidth}
  \centering
  \includegraphics[width=\textwidth]{frequency_lookback_study3}
  \captionsetup{justification=centering}
  \captionof{figure}{Frequency of lookback of the participant on study 3}
  \label{fig:freq_study3}
\end{minipage}
\end{figure}

\begin{figure}[!h]
\centering
\begin{minipage}{.5\textwidth}
  \centering
  \includegraphics[width=\textwidth]{lookback_and_reading_time_study1}
  \captionsetup{justification=centering}
  \captionof{figure}{Average time each participant looking for the answer in study 1}
  \label{fig:aveTime_study1}
\end{minipage}%
\begin{minipage}{.5\textwidth}
  \centering
  \includegraphics[width=\textwidth]{lookback_and_reading_time_study3}
  \captionsetup{justification=centering}
  \captionof{figure}{Average time of all the participants looking for the answer in study 3}
  \label{fig:aveTime_study3}
\end{minipage}
\end{figure}

\subsection{Discussion}
The result of the bar chart \ref{fig:lookingAnswer_lookback}  shows that the paricipant will more likely to experience prospective memory error if they read longer than the average time
of the participant who do not experience memory error. The participant probably miss the answer or read some interesting information which
make them read longer, then they experience lost intention \cite{Reason1984} as a result they experience  prospective memory error.
We can argue that the participant experience the event boundary which is moving from the android application to the answer page for looking to the answer.
The time spent on reading represent the level of immersion people have on after the event boundary. So the type and the level of immersion of the task after the event
boundary hold a signifact factor on deciding if the person willl experience failure of prospective memory or not.

To understand the effect of physical transition through event boundary we try to investigate the frequency of prospective memory error between first on the third study.
Bar chart \ref{fig:freq_study1} and \ref{fig:freq_study3} shows that the participant on the third study forget the question more frequently than the first study. However,
it cannot show strong correlation between physically moving through another room with prospective memory failure phenomena.
Because the sample is very small so we cannot make any solid conclusions on whether the physical transtition of the event boundary will increase the probability of a person to experience
prospective memory error. However, by looking at bar charts \ref{fig:aveTime_study1} and \ref{fig:aveTime_study3} we can see that the physical transtition result on the longer time for people to read and find the answer.
This shows that the physical transition decrease the capability of cognitive ability while doing this experiment.
