
\chapter{Literature Review}
\section{Prospective memory and retrospective memory}

Tasks such as buying milk in a supermarket on the way to work action, turning off the oven and taking a medication are categorized as a prospective memory task. Prospective memory is used constantly in everyday activity \citep{gruneberAndMoris1978}, \citep{cohen1989}.
There are a lot of definition about prospective memory, but generally  a prospective memory is defined as remembering to carry out planned actions at a particular time in the future without being instructed to do so \citep{mcdaniel2007prospective}; \citep{GROOT2002}. While task such as answering the question on an exam or remembering the person name on the party is categorized as a retrospective memory task.
Retrospective memory involves remembering events, words, and so on from the past typically
when deliberating trying to do so.

% The important difference between retrospective and prospective is prospective memory involve remembering to carry out intended actions without being instructed to do so.
% \todo[inline]{keknya paragraph dibawah ini perlu di ubah}


% Remembering in prospective memory is difficult because it require the interuption of flow of thought when we are on ongoing activity, in retrospective memory this interuption is externaly promted e.g question during exam. this made retrospective memory driven by high perceptual information while retrospective  while prospective memory perform on low information content. (McDaniel et al., 1998)

According to \cite{BaddeleyWilkins1983}, it's very hard to differentiate between prospective memory and retrospective memory because there is no clear cut between them. for example, To remember to call your father, you should able to recall his number and how to use the phone, and not call him while he watches a football match. \cite{brandimonte1996prospective} call this as retrospective component of a prospective memory task.
\cite{CockburnJ.1995Tiip} stated that content of the information is similar to both memory type but the essential difference is prospective memory require memory for intention and the cue for retrieval has to be self-initiated.
\cite{GuynnMelissaJ.1998PMWR} also state that retrospective memory is driven by low information content while retrospective memory is driven by high perceptual information, such as question during an exam.

Furthermore, Remembering only the retrospective memory component of a prospective memory task will not produce successful prospective memory. In fact, numerous prospective memory failures happened because the failure of remembering the prospective memory component
\citep{einsteindGuynn1992}. Interestingly, the component of retrospective memory sometimes forgotten in a simple prospective memory task, for instance when we walk to the kitchen and sometimes forget what we are intended to do there \citep{brandimonte1996prospective}.

% Retrospective memory focus on remembering the content of information like or what we know about something, while the prospective memory focus on implementing the delayed intention or when to do something (Baddeley, 1997).

\section{Cognitive process of prospective memory}
%\todo[inline]{this last paragraph is actually sampah, our focus is what makes us forget not what make us remember}

% This memory division is based on the cognitive processes and Neuroanatomy bases which determine these types of memory . (Maylor, 1995) (Bieriet al., 2014) (Tierney et al., 2016)

Some researcher believes that prospective memory proceeds  through encoding, retention, retrieval, execution and evaluation phase.
According to \cite{inside1996prospective} In the Encoding phase, the \textit{when}(retrieval criterion), \textit{what}(action to be performed) and \textit{that}(intent or decision to act) are encoded. Then this intention representation must be retained until the opportunity to fulfill the intention occurs. this delayed can vary from a second to a week. \cite{EinsteinGillesO.1990NAaP} categorize  retrieval process, event-based or time-based retrieval. On the event based retrieval, the retrieval happens if there is a particular event or physical stimulus that associated with the intention. for example telling a message when you meet your college. On the other hand, time-based retrieval require execution of action after a certain time  \citep{inside1996prospective};   \citep{Mcgann2002}.
Therefore, successful prospective remembering can be described as a process that support the actualization of delayed attention and the associated action, and it is strongly as associated with control or coordination of future action \citep{inside1996prospective}.

\section{prospective memory error}

Prospective memory error is defined as a failure to do a planned action at some point or at a particular event in the future. \cite{Kliegel1984} state that prospective memory failure is the most frequent memory failure in everyday life.  The ability to remember the planned action is a critical factor in human functioning. The consequence of a failure of prospective memory can be trivial for example forgetting to buy some milk on the way home from work. But it can also have severe consequences, for example, the doctor forget to took the scalpel from his patient after an operation. In fact, \cite{Shorrock2005} reported that 38\% of accidents on the traffic controllers in the UK due to memory error involves the failure of prospective memory.


Many researchers has different view on the prospective memory error and what cause it to happen.

\cite{LiaKvavilashviliAndJudiEllis} try to categorize and differentiate a various kind of memory error with a prospective memory error. They claim that \textit{action-slip}\citep{HeckhausenHeinz1990IAaA}, \textit{actions-not-as-planned} \citep{Reason1979-REAANA-2} and \textit{absent-minded error} \citep{cohen2008memory} should not constitute as a prospective memory error. These errors happens because the failure that occurs during the execution or performance of the intended action. for example, in absent minded error people lose the context of an intention and carry out an unintended action instead of the intended one. In contrast, prospective memory is focused on the failure to retrieve intended action.
While \cite{10.1371/journal.pone.0074447} argue that these type of error should be considered as part of prospective memory error because prospective memory contains some element of retrospective memory such as the context of intention. Moreover, \cite{Reason1984} explained further on how the element of memory; context, intention and attention influence prospective memory error.  In addition, \cite{Cockburn1994} argued that stress and anxiety make a person to experience absent minded error hence make a prospective memory error.  and \cite{Scullin2012} found older people tend to make more error than younger people on a prospective memory test.

\section{Prospective memory and intention}
Because prospective memory refers to remembering intentions so it would be better to have a good understanding of intention first. For example to understand the nature of intention and its phenomena,the category of intention and how it related to everyday activities and what happen to intention during prospective memory error. The explanation of these question maybe gives us more understanding about the correlation between intention and prospective memory error.

\cite{LiaKvavilashviliAndJudiEllis}, \cite{gauld1977human} define an intention as a person's readiness to act in a certain way in the future. what has to be done and when to be done should be defined clearly.

\cite{searle1983intentionality} distinguished intention into two types, prior-intention and intention-in-action. A prior intention is an intention that is defined prior to action, while intention-in-action is a spontaneous action, for example going to the toilet when you need to urinate. A prior intention is always occurred as a result of conscious decision to act in a certain way \citep{Heckhausen1985-HECFWT}. Furthermore, \cite{gauld1977human} categorized prior intention into two categories, delayed intentions and immediate intention.The delayed intention is a postponed intention that will be executed at some point in the future, and when a person begins to carry out their prior intention immediately after a decision has been made or after they see a particular cue for the intention.


The difficulty of retrieval of the delayed intention make persons miss the prearranged moment or cues, and this make people fail to remember. Even though people able to retrieve the delayed intention, but when the intention is initiated and transformed into an immediate intention,  people can still lose their intention and prospective memory error occurs.
Furthermore, \cite{Reason1984} explain how a change in the intention make people experience memory error by categorizing two phenomena called \textit{detached intention} and \textit{lost intention}.

\subsubsection{Detached intention}

Detached intention happen if the content of the intention detached from its proper setting and misapplied to something other than its original intention. For example, the case when a person switches off the television instead of the oven. the possible explanation for this phenomena is because the intention is not framed completely, probably because the focus of attention was claimed on other things (this will be explained further on the attention section). Another explanation is that the intention was framed completely but has not sufficient level to be retained until the moment of execution. Another explanation is there is another combination of intention that has similar action and trigger from another object which similar kind of action is appropriate \citep{Reason1984}.


\subsubsection{Lost of intention}

While detached intentions happened because of partial failure of the intention and retention system. Lost intention
is a complete breakdown at one or more of the stage of formulation, encoding, storage, or retrieval of the intention.
One typical case is when an intention is lost during the retrieval phase. for instance when a person walks into a room and become aware that he/she can't recall the original intention of the activity \citep{Reason1984}.


\section{Prospective memory and attention}

% Our daily life are strewn with such trifling and usually inconsequential blunders

When we accidentally put our phone in the fridge instead of our food or when we pour the second kettle of water into a freshly made coffee. These slips of action frequently occur as the result of misdirected or diminished attention \cite{Reason1984}
James defines attention as  "the taking possession by the mind, in clear and vivid form, of one out of what seem several simultaneously possible objects or trains of thought".
There is a minimum degree of attentional involvement is necessary to ensure the right execution of the sequence of attentions, and to avoid someone make a mistake due to some kind of attentional failure.

\cite{Reason1984} define attentions as the gatekeeper of consciousness. This definition marks an important role of attention and consciousness in the performance of delayed intention on prospective memory. A person must be conscious of the plan to perform an action. To be conscious about it, the plan should be the focus of attention. the attention should be kept at the encoding phase when the action is planned and at retrieval when the action is performed.

But error can also be occur when a person is putting too much attention on the ongoing activity, for example, running down the stair two at a time, this should be an automatic activity but when a person does it with too much attention then it can be very disruptive.

Moreover, dividing attention is also assumed to reduce the contribution of a controlled process, thereby
reducing performance on a memory test that involves conscious recollection \citep{Jacoby1989}.
Some previous study also shows that there was a substantial reduction in prospective memory performance when attention is divided \citep{McDaniel1998}
\citep{10.1371/journal.pone.0074447}.


\section{Prospective memory error at event Boundaries}

We walk to the park, read a book, watch a movie and do numerous things, one after another. These stream of actions consist of events. How we split up these stream of action into events and stored them into memory influence how we think and what to remember. Memory and cognition are heavily influenced by event and how a person structures them \citep{Radvansky2012}. \cite{Radvansky2011} introduce an event model which is a mental model that captures the content and structure of an event that people experience.

\cite{Radvansky2012} also suggest that when persons make a cognitive transition from one event to another, they will experience an event boundary. Such transitions can be a change in location, a causal break, the introduction of a new activity, and so on, as long as they involve a shift from one event to another.  On some condition, event boundaries can disrupt memory. When people experience event boundaries they mentally update their event model. \cite{Radvansky2010} investigate about this phenomena in the reading experiment and shows that the updating effect of a mental model increases the reading time of a sentence. the increase reflects increase on cognitive effort need for the updating.


% in a series of studies we had done (Radvansky \& Copeland,
% 2006b;  Radvansky,  Krawietz,  \&  Tamplin,  2011;  Radvansky,
% Tamplin, \& Krawietz, 2010)  We  found
% that people took longer and made more errors when there was
% an event shift than when there was not. In other words, walking through doorways caused forgetting.

Furthermore, \cite{Radvansky2010} found that when people pass through the doorway to move from
one location to another, they forget more information that if they do not make such a shift. This
effect is similar to the result from other research in text comprehension that shows that shift
in location decline memory performance \citep{Curiel2002}; \citep{Haenggi1995}; \citep{Radvansky2010}; \citep{Radvansky2003}.
Moreover, that study also showed that if people travelled through two doorways, they were more
likely to forget than if they had travelled through only one.

\cite{Kurby2008} and \cite{Swallow2009} proposed event segmentation theory
which explains the correlation between memory and event. The theory state that during the experience of an event, when event boundaries are identified, people segmented information into separate event models and then stored it into memory.

All these previous research result in event horizon model proposed by \cite{Radvansky2012}
The event horizon model use event segmentation theory and explained that when an event is
segmented and stored as event model, it declines in availability and become deactivated. And
as person experience event boundaries, a new event model is created in working memory. The
active event model that is currently at the working memory is foregrounded which make it easier
to retrieve ,and an available processing capacity is directed to it.

The presentation of a memory cue causes both models that contain target information to
be activated this result on competition and interference, which slows down response times and
increases error rates. This is why returning to a previous room does not improve memory for
objects that were encountered there, and why passing through two doorways makes memory even
worse than does passing through one \citep{Radvansky2011}.


\section{Previous research}
Lisa. M. Stevenson and Richard A. Carlson from Pennsylvania State University conducted an
experiment on the failure of prospective memory. The experiment conducted three studies,
On the experiment, each participant used a mobile phone to answer eight trivia questions that randomly selected from three different topics (movies and TV, geography or Penn state trivia). On each question, an embedded link is presented, and the participant is instructed to find the answer on the web page. Subsequently, the participant is asked questions to assess their prospective memory. The experiments consist of three studies and each study
answer different hypothesis.

The first study aimed to assess whether the prospective memory failure happened when
a participant uses the smart phone.
two question at each time are presented, and 63 participant have participated on this study.
This based on a phenomenon when people clicked a link
on a website, and then forget what they are looking up. The study shows that about 75\% of the
participant experience the failure of perspective memory which shows that the failure of prospective memory happened even when a person is using a smartphone.

The second study aimed to evaluate the
effect of multiple attention  with the perspective memory. The number of intention is represented
as a number of questions being asked at a time. The study presents the participant with one or
two questions at a time. The result shows that the failure of prospective memory more likely to
happen when two questions are asked. This means that the amount of intention is an important factor of prospective memory (intentional loads).

Lastly, the third study aimed to evaluate the effect of event boundaries memory (location
shifting) on prospective memory. 84 students participate in this study. One question at a time
is asked, and the participant is instructed to move within a room, between rooms or stay seated.
The study shows that the participants do not experience the failure or improvement of perspective
memory when there has been a shift in location.
